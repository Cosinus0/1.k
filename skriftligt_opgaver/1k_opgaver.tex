\documentclass[10pt,a4paper]{article}
\usepackage[utf8x]{inputenc}
\usepackage[T1]{fontenc}
\usepackage{ucs}
\usepackage[english]{babel}
\usepackage{amsmath}
\usepackage{amsfonts}
\usepackage{amssymb}
\usepackage{graphicx}
\setlength\parindent{0pt}
\begin{document}
	\title{1k Opgaver sæt}
	\maketitle
	
\section{Funktioner}
\subsection{Linear Funktioner}
\begin{itemize}
	\item Ved at bruge to punkter $(6,12)$ og $(24,18)$, find forskiften til den linear funktion.
	\item Hvad er Dm(f) og Vm(f) af den linear funktion.
	\item \begin{tabular}{|c|c|c|c|c|c|}
		\hline 
		x & 0 & 2 & 6 &  & 24 \\ 
		\hline 
		y &  &  & 12 & 13 & 18 \\ 
		\hline 
	\end{tabular}
	\item Tegn den linear funktion i hånden. 
\end{itemize}

\subsection{Karakterisering af en funktion}

\begin{equation*}
f(x)=2x^3-3x^2+1
\end{equation*}

Bestem monotoniforhold og global max og min.

\subsection{Regning med funktioner}

\begin{align*}
f(x)=x^3+2x^2-8+10\\
g(x)=4x^3-5x^2+11\\
h(x)=16x^2-2x\\
i(x)=8x-1
\end{align*}

\textbf{Bestem:}
\begin{itemize}
	\item f(x)+g(x)
	\item g(x)-f(x)
	\item f(x)h(x)
	\item $\frac{h(x)}{i(x)}$ (Hint: faktorisere h(x) så du har $h(x) = \text{et faktor} \cdot i(x)$)
\end{itemize}

\subsection{Sammesat Funktioner:}

\begin{align*}
f(x)=x^2+5\\
g(x)=2x-1\\
h(x)=\sqrt{x}
\end{align*}

\textbf{Bestem:}
\begin{itemize}
	\item f(g(x))
	\item g(f(x))
	\item f(h(x))
	\item h(f(x))
\end{itemize}

\subsection{Omvendt Funktioner:}
\textbf{Opgave 174} i opgavebogen.

\section{Rødder og Potenser}
\textbf{Bestem:}
\begin{itemize}
	\item $9p^2-p^2+7p-10p$
	\item $7k^2+k^3-6k+5-9k^3+8k-7$
	\item $4a \cdot 5b \cdot 6c$
	\item $3x^3 \cdot 9x$
	\item $8t^3 \cdot 4t^4$
	\item $3x^5y^3 \cdot 2xy^4$
	\item $\frac{10d}{2d}$
	\item $\frac{6t^6}{2t^2}$
	\item $\frac{14a^3b^2c}{2abc}$
	\item $\frac{8m^8n^6}{12m^5n^9}$
	\item $(4a^5)^2$
	\item $(x^3y^4)^3$
	\item $7(c^3d^7)^2$
	\item $(3a^2b^3c^4)^4$
\end{itemize}

\section{Logaritmefunktioner}
\textbf{Opgaver 309, 312 og 316} i opgavebogen.

\section{Annuiteter}
\textbf{Opgaver 436, 440, og 448} i opgavebogen.

\section{Eksponentialfunktioner}
\textbf{Opgaver 542, 586 og 598} i opgavebogen.

\section{Potensfunktioner}
\textbf{Opgaver 623 og 626} i opgavebogen.

\section{Deskriptiv Statistik}
\begin{itemize}
\item Hvad er definition for ugrupperede og grupperede observationer?
\item \textbf{Opgave 1112} i opgavebogen (Bestem også:)
\begin{itemize}
	\item Expection value
	\item Variationsbredden
	\item Medianen
	\item Varians
	\item Spredning
\end{itemize}
\item \textbf{Opgave 1121} i opgavebogen. 	
\end{itemize}
\end{document}